% Options for packages loaded elsewhere
\PassOptionsToPackage{unicode}{hyperref}
\PassOptionsToPackage{hyphens}{url}
%
\documentclass[
]{article}
\title{Exam}
\author{Andrei Nistor}
\date{1/3/2022}

\usepackage{amsmath,amssymb}
\usepackage{lmodern}
\usepackage{iftex}
\ifPDFTeX
  \usepackage[T1]{fontenc}
  \usepackage[utf8]{inputenc}
  \usepackage{textcomp} % provide euro and other symbols
\else % if luatex or xetex
  \usepackage{unicode-math}
  \defaultfontfeatures{Scale=MatchLowercase}
  \defaultfontfeatures[\rmfamily]{Ligatures=TeX,Scale=1}
\fi
% Use upquote if available, for straight quotes in verbatim environments
\IfFileExists{upquote.sty}{\usepackage{upquote}}{}
\IfFileExists{microtype.sty}{% use microtype if available
  \usepackage[]{microtype}
  \UseMicrotypeSet[protrusion]{basicmath} % disable protrusion for tt fonts
}{}
\makeatletter
\@ifundefined{KOMAClassName}{% if non-KOMA class
  \IfFileExists{parskip.sty}{%
    \usepackage{parskip}
  }{% else
    \setlength{\parindent}{0pt}
    \setlength{\parskip}{6pt plus 2pt minus 1pt}}
}{% if KOMA class
  \KOMAoptions{parskip=half}}
\makeatother
\usepackage{xcolor}
\IfFileExists{xurl.sty}{\usepackage{xurl}}{} % add URL line breaks if available
\IfFileExists{bookmark.sty}{\usepackage{bookmark}}{\usepackage{hyperref}}
\hypersetup{
  pdftitle={Exam},
  pdfauthor={Andrei Nistor},
  hidelinks,
  pdfcreator={LaTeX via pandoc}}
\urlstyle{same} % disable monospaced font for URLs
\usepackage[margin=1in]{geometry}
\usepackage{color}
\usepackage{fancyvrb}
\newcommand{\VerbBar}{|}
\newcommand{\VERB}{\Verb[commandchars=\\\{\}]}
\DefineVerbatimEnvironment{Highlighting}{Verbatim}{commandchars=\\\{\}}
% Add ',fontsize=\small' for more characters per line
\usepackage{framed}
\definecolor{shadecolor}{RGB}{248,248,248}
\newenvironment{Shaded}{\begin{snugshade}}{\end{snugshade}}
\newcommand{\AlertTok}[1]{\textcolor[rgb]{0.94,0.16,0.16}{#1}}
\newcommand{\AnnotationTok}[1]{\textcolor[rgb]{0.56,0.35,0.01}{\textbf{\textit{#1}}}}
\newcommand{\AttributeTok}[1]{\textcolor[rgb]{0.77,0.63,0.00}{#1}}
\newcommand{\BaseNTok}[1]{\textcolor[rgb]{0.00,0.00,0.81}{#1}}
\newcommand{\BuiltInTok}[1]{#1}
\newcommand{\CharTok}[1]{\textcolor[rgb]{0.31,0.60,0.02}{#1}}
\newcommand{\CommentTok}[1]{\textcolor[rgb]{0.56,0.35,0.01}{\textit{#1}}}
\newcommand{\CommentVarTok}[1]{\textcolor[rgb]{0.56,0.35,0.01}{\textbf{\textit{#1}}}}
\newcommand{\ConstantTok}[1]{\textcolor[rgb]{0.00,0.00,0.00}{#1}}
\newcommand{\ControlFlowTok}[1]{\textcolor[rgb]{0.13,0.29,0.53}{\textbf{#1}}}
\newcommand{\DataTypeTok}[1]{\textcolor[rgb]{0.13,0.29,0.53}{#1}}
\newcommand{\DecValTok}[1]{\textcolor[rgb]{0.00,0.00,0.81}{#1}}
\newcommand{\DocumentationTok}[1]{\textcolor[rgb]{0.56,0.35,0.01}{\textbf{\textit{#1}}}}
\newcommand{\ErrorTok}[1]{\textcolor[rgb]{0.64,0.00,0.00}{\textbf{#1}}}
\newcommand{\ExtensionTok}[1]{#1}
\newcommand{\FloatTok}[1]{\textcolor[rgb]{0.00,0.00,0.81}{#1}}
\newcommand{\FunctionTok}[1]{\textcolor[rgb]{0.00,0.00,0.00}{#1}}
\newcommand{\ImportTok}[1]{#1}
\newcommand{\InformationTok}[1]{\textcolor[rgb]{0.56,0.35,0.01}{\textbf{\textit{#1}}}}
\newcommand{\KeywordTok}[1]{\textcolor[rgb]{0.13,0.29,0.53}{\textbf{#1}}}
\newcommand{\NormalTok}[1]{#1}
\newcommand{\OperatorTok}[1]{\textcolor[rgb]{0.81,0.36,0.00}{\textbf{#1}}}
\newcommand{\OtherTok}[1]{\textcolor[rgb]{0.56,0.35,0.01}{#1}}
\newcommand{\PreprocessorTok}[1]{\textcolor[rgb]{0.56,0.35,0.01}{\textit{#1}}}
\newcommand{\RegionMarkerTok}[1]{#1}
\newcommand{\SpecialCharTok}[1]{\textcolor[rgb]{0.00,0.00,0.00}{#1}}
\newcommand{\SpecialStringTok}[1]{\textcolor[rgb]{0.31,0.60,0.02}{#1}}
\newcommand{\StringTok}[1]{\textcolor[rgb]{0.31,0.60,0.02}{#1}}
\newcommand{\VariableTok}[1]{\textcolor[rgb]{0.00,0.00,0.00}{#1}}
\newcommand{\VerbatimStringTok}[1]{\textcolor[rgb]{0.31,0.60,0.02}{#1}}
\newcommand{\WarningTok}[1]{\textcolor[rgb]{0.56,0.35,0.01}{\textbf{\textit{#1}}}}
\usepackage{longtable,booktabs,array}
\usepackage{calc} % for calculating minipage widths
% Correct order of tables after \paragraph or \subparagraph
\usepackage{etoolbox}
\makeatletter
\patchcmd\longtable{\par}{\if@noskipsec\mbox{}\fi\par}{}{}
\makeatother
% Allow footnotes in longtable head/foot
\IfFileExists{footnotehyper.sty}{\usepackage{footnotehyper}}{\usepackage{footnote}}
\makesavenoteenv{longtable}
\usepackage{graphicx}
\makeatletter
\def\maxwidth{\ifdim\Gin@nat@width>\linewidth\linewidth\else\Gin@nat@width\fi}
\def\maxheight{\ifdim\Gin@nat@height>\textheight\textheight\else\Gin@nat@height\fi}
\makeatother
% Scale images if necessary, so that they will not overflow the page
% margins by default, and it is still possible to overwrite the defaults
% using explicit options in \includegraphics[width, height, ...]{}
\setkeys{Gin}{width=\maxwidth,height=\maxheight,keepaspectratio}
% Set default figure placement to htbp
\makeatletter
\def\fps@figure{htbp}
\makeatother
\setlength{\emergencystretch}{3em} % prevent overfull lines
\providecommand{\tightlist}{%
  \setlength{\itemsep}{0pt}\setlength{\parskip}{0pt}}
\setcounter{secnumdepth}{-\maxdimen} % remove section numbering
\ifLuaTeX
  \usepackage{selnolig}  % disable illegal ligatures
\fi

\begin{document}
\maketitle

\#1. North America Rodents. \#\#Dataset data/surveys.csv -rodents
sightings in North America from 1977 to 2002 \#\#Dataset
data/species.csv -species acronyms and their Genus

\hypertarget{a-join-the-two-datasets}{%
\subsection{a) Join the two datasets}\label{a-join-the-two-datasets}}

\begin{Shaded}
\begin{Highlighting}[]
\NormalTok{surveys }\OtherTok{\textless{}{-}}\NormalTok{ readr}\SpecialCharTok{::}\FunctionTok{read\_csv}\NormalTok{(}\StringTok{\textquotesingle{}data/surveys.csv\textquotesingle{}}\NormalTok{)}
\end{Highlighting}
\end{Shaded}

\begin{verbatim}
## New names:
## * `` -> ...1
\end{verbatim}

\begin{verbatim}
## Rows: 30738 Columns: 10
\end{verbatim}

\begin{verbatim}
## -- Column specification --------------------------------------------------------
## Delimiter: ","
## chr (2): species_id, sex
## dbl (8): ...1, record_id, month, day, year, plot_id, hindfoot_length, weight
\end{verbatim}

\begin{verbatim}
## 
## i Use `spec()` to retrieve the full column specification for this data.
## i Specify the column types or set `show_col_types = FALSE` to quiet this message.
\end{verbatim}

\begin{Shaded}
\begin{Highlighting}[]
\FunctionTok{spec}\NormalTok{(surveys)}
\end{Highlighting}
\end{Shaded}

\begin{verbatim}
## cols(
##   ...1 = col_double(),
##   record_id = col_double(),
##   month = col_double(),
##   day = col_double(),
##   year = col_double(),
##   plot_id = col_double(),
##   species_id = col_character(),
##   sex = col_character(),
##   hindfoot_length = col_double(),
##   weight = col_double()
## )
\end{verbatim}

\begin{Shaded}
\begin{Highlighting}[]
\NormalTok{species }\OtherTok{\textless{}{-}}\NormalTok{ readr}\SpecialCharTok{::}\FunctionTok{read\_csv}\NormalTok{(}\StringTok{\textquotesingle{}data/species.csv\textquotesingle{}}\NormalTok{)}
\end{Highlighting}
\end{Shaded}

\begin{verbatim}
## Rows: 54 Columns: 4
\end{verbatim}

\begin{verbatim}
## -- Column specification --------------------------------------------------------
## Delimiter: ","
## chr (4): species_id, genus, species, taxa
\end{verbatim}

\begin{verbatim}
## 
## i Use `spec()` to retrieve the full column specification for this data.
## i Specify the column types or set `show_col_types = FALSE` to quiet this message.
\end{verbatim}

\begin{Shaded}
\begin{Highlighting}[]
\FunctionTok{spec}\NormalTok{(species)}
\end{Highlighting}
\end{Shaded}

\begin{verbatim}
## cols(
##   species_id = col_character(),
##   genus = col_character(),
##   species = col_character(),
##   taxa = col_character()
## )
\end{verbatim}

\begin{Shaded}
\begin{Highlighting}[]
\NormalTok{rodents }\OtherTok{\textless{}{-}}\NormalTok{ surveys }\SpecialCharTok{\%\textgreater{}\%} \FunctionTok{left\_join}\NormalTok{(species, }\AttributeTok{by =} \FunctionTok{c}\NormalTok{(}\StringTok{\textquotesingle{}species\_id\textquotesingle{}} \OtherTok{=} \StringTok{\textquotesingle{}species\_id\textquotesingle{}}\NormalTok{))}
\FunctionTok{spec}\NormalTok{(rodents)}
\end{Highlighting}
\end{Shaded}

\begin{verbatim}
## cols(
##   ...1 = col_double(),
##   record_id = col_double(),
##   month = col_double(),
##   day = col_double(),
##   year = col_double(),
##   plot_id = col_double(),
##   species_id = col_character(),
##   sex = col_character(),
##   hindfoot_length = col_double(),
##   weight = col_double()
## )
\end{verbatim}

\begin{enumerate}
\def\labelenumi{\alph{enumi})}
\setcounter{enumi}{1}
\tightlist
\item
  Present the 5 rodent species having the highest mean weight in a table
  showing species, mean weight and mean hindfoot length as in the
  example below.
\end{enumerate}

\begin{Shaded}
\begin{Highlighting}[]
\NormalTok{  mean\_weight }\OtherTok{\textless{}{-}}\FunctionTok{mean}\NormalTok{(surveys}\SpecialCharTok{$}\StringTok{\textasciigrave{}}\AttributeTok{weight}\StringTok{\textasciigrave{}}\NormalTok{)  }
\NormalTok{  mean\_length }\OtherTok{\textless{}{-}}\FunctionTok{mean}\NormalTok{(surveys}\SpecialCharTok{$}\StringTok{\textasciigrave{}}\AttributeTok{hindfoot\_length}\StringTok{\textasciigrave{}}\NormalTok{)}
\end{Highlighting}
\end{Shaded}

\begin{Shaded}
\begin{Highlighting}[]
\NormalTok{rodents }\SpecialCharTok{\%\textgreater{}\%} 
  \FunctionTok{group\_by}\NormalTok{(species) }\SpecialCharTok{\%\textgreater{}\%} 
  \FunctionTok{summarize}\NormalTok{(}\StringTok{"Mean Weight [g]"} \OtherTok{=} \FunctionTok{mean}\NormalTok{(weight),}\StringTok{"Mean Hindfoot Length [mm]"}\OtherTok{=}\FunctionTok{mean}\NormalTok{(hindfoot\_length)) }\SpecialCharTok{\%\textgreater{}\%} 
  \FunctionTok{top\_n}\NormalTok{(}\DecValTok{5}\NormalTok{, }\StringTok{\textasciigrave{}}\AttributeTok{Mean Weight [g]}\StringTok{\textasciigrave{}}\NormalTok{) }\SpecialCharTok{\%\textgreater{}\%}
  \FunctionTok{arrange}\NormalTok{(}\FunctionTok{desc}\NormalTok{(}\StringTok{\textasciigrave{}}\AttributeTok{Mean Weight [g]}\StringTok{\textasciigrave{}}\NormalTok{)) }\SpecialCharTok{\%\textgreater{}\%}
\NormalTok{  knitr}\SpecialCharTok{::}\FunctionTok{kable}\NormalTok{()}
\end{Highlighting}
\end{Shaded}

\begin{longtable}[]{@{}lrr@{}}
\toprule
species & Mean Weight {[}g{]} & Mean Hindfoot Length {[}mm{]} \\
\midrule
\endhead
albigula & 158.72371 & 32.25048 \\
spectabilis & 120.21668 & 49.99260 \\
hispidus & 64.84906 & 28.05031 \\
fulviventer & 59.12500 & 26.70000 \\
ochrognathus & 55.37500 & 25.60000 \\
\bottomrule
\end{longtable}

\hypertarget{c-recreate-the-plot.}{%
\subsection{c) Recreate the plot.}\label{c-recreate-the-plot.}}

It seems like the plot is a scatter type.

\begin{Shaded}
\begin{Highlighting}[]
\NormalTok{bigrodents}\OtherTok{\textless{}{-}} \FunctionTok{filter}\NormalTok{(rodents,genus}\SpecialCharTok{==}\StringTok{\textquotesingle{}Dipodomys\textquotesingle{}}\NormalTok{) }
  \FunctionTok{ggplot}\NormalTok{( }\AttributeTok{data =}\NormalTok{ bigrodents) }\SpecialCharTok{+}
  \FunctionTok{geom\_point}\NormalTok{(}\AttributeTok{mapping=}\FunctionTok{aes}\NormalTok{( }\AttributeTok{x =}\NormalTok{ weight, }\AttributeTok{y =}\NormalTok{ hindfoot\_length, }\AttributeTok{color =}\NormalTok{ species)) }\SpecialCharTok{+}
  \FunctionTok{labs}\NormalTok{(}\AttributeTok{title =} \StringTok{\textquotesingle{}Dipodomys  Kangaroo Rats\textquotesingle{}}\NormalTok{,}
       \AttributeTok{subtitle =}\StringTok{\textquotesingle{}Caught in years 1980−1989\textquotesingle{}}\NormalTok{,}
       \AttributeTok{x =} \StringTok{\textquotesingle{}Hindfoot Length [mm]\textquotesingle{}}\NormalTok{,}
       \AttributeTok{y =} \StringTok{\textquotesingle{}Weight [g]\textquotesingle{}}\NormalTok{)}\SpecialCharTok{+}
  \FunctionTok{xlim}\NormalTok{(}\DecValTok{0}\NormalTok{,}\DecValTok{50}\NormalTok{) }\SpecialCharTok{+} 
  \FunctionTok{ylim}\NormalTok{(}\DecValTok{0}\NormalTok{,}\DecValTok{150}\NormalTok{)}
\end{Highlighting}
\end{Shaded}

\begin{verbatim}
## Warning: Removed 4346 rows containing missing values (geom_point).
\end{verbatim}

\includegraphics{ANDREI_NISTOR_STATISTICS_EXAM--1-_files/figure-latex/unnamed-chunk-6-1.pdf}

\begin{enumerate}
\def\labelenumi{\alph{enumi})}
\setcounter{enumi}{3}
\item
  Describe the plot. First of all,the plot illustrates the size of the
  rodents from the genus Dipodomys, specifically their weight and
  hindfoot length. The first thing I noticed when analyzing the dataset
  was that those rodents are mainly the ones that are bigger in size
  than rodents from other genus, with spectabilis being the heaviest
  reaching even above 150g, at almost 200g, and with a hindfoot length
  close to 60 mm. Furthermore, this result gives an insight into why
  they were named like that ,thanks to their spectacular sizes.
  Although, the ordii and the merriami are smaller sometimes they can
  reach a hindfoot length closer to the spectabilis even though they
  might be 3 times lighter in weight. For example, in the dataset we can
  see the longest hindfoot recorded was that of an ordii with 64 mm.
  Based on my assesment of the dataset, I can confidently say that the
  Dipodomys are mainly the bigger rodents , along with albigula from the
  Neotoma genus, which seems to be the heaviest rodent, but lacks the
  hindfoot length of the Dipodomys genus. Thus, from our dataset we can
  conclude that the Dipodomys are the rodents with the longest hindfoot
  length.
\item
  Kangaroo Rats (genus Dipodomys) are small rodents moving similarily to
  kangaroos using jumping.steps. Is the mean male hindfoot length
  different between ordii and merriami species? Comment on the results
\end{enumerate}

Analyzing the dataset , I noticed that the biggest hind foot recorder
was by a female ordii with 64mm, which is very impressive compared to
her lighter build of 35g . It seems like the male merriami does not go
above 40 mm in hind foot length, while the ordii seem to vary more ,
with the highest hind foot recorder by a male being 58 mm.

\#2. Medical students smoking habits

A study was conducted on various Medical Universities within Germany and
Hungary. The students were asked about their smoking habits. 2883
students took part, 44\% of them were German, 36\% were Hungarian and
20\% other nationalities. The table below lists number of students per
nationality declaring daily smoking habit.

\#\#a) Are the proportions of nationalities within smoking students a
true representation of proportions of the whole student body? Conduct a
suitable test to check this hypothesis

-2883 students took part -44\% of them were German, 36\% were Hungarian
and 20\% other nationalities. -91 germans,78 hungarians, 51
multinational

\begin{Shaded}
\begin{Highlighting}[]
\NormalTok{smoking }\OtherTok{\textless{}{-}} \FunctionTok{tribble}\NormalTok{(}\SpecialCharTok{\textasciitilde{}}\StringTok{\textquotesingle{}Nationality\textquotesingle{}}\NormalTok{, }\SpecialCharTok{\textasciitilde{}}\StringTok{\textquotesingle{}n\textquotesingle{}}\NormalTok{, }
\StringTok{\textquotesingle{}German\textquotesingle{}}\NormalTok{, }\DecValTok{91}\NormalTok{,}
\StringTok{\textquotesingle{}Hungarian\textquotesingle{}}\NormalTok{, }\DecValTok{78}\NormalTok{,}
\StringTok{\textquotesingle{}Multinational\textquotesingle{}}\NormalTok{, }\DecValTok{51}\NormalTok{)}
\FunctionTok{print}\NormalTok{(smoking)}
\end{Highlighting}
\end{Shaded}

\begin{verbatim}
## # A tibble: 3 x 2
##   Nationality       n
##   <chr>         <dbl>
## 1 German           91
## 2 Hungarian        78
## 3 Multinational    51
\end{verbatim}

\hypertarget{i-will-start-by-checking-if-i-can-apply-the-central-limit-theorem}{%
\subsubsection{I will start by checking if i can apply the central limit
theorem:}\label{i-will-start-by-checking-if-i-can-apply-the-central-limit-theorem}}

-1st condition: if the data is independent.We have to assume the data is
independent. -2st condition : check if sample size larger than 30. It is
not so we cannot do CLT.

Making the hypothesis with the help of null hypothesis: H0: There is no
relation between the percentage of smoking students and the percentage
of nationalities within the student body. HA: : There is an accurate
representation of the relation between the smoking students and the
percentage of nationalities within the student body. This proves that
smoking habits can be influenced by one's nationality.

Now we have to choose an alpha significance value. I will choose alpha
signivicant value= 0.05 We can now run the function for the Chi-squared
test.

\begin{Shaded}
\begin{Highlighting}[]
\NormalTok{smoking }\SpecialCharTok{\%\textgreater{}\%} \FunctionTok{select}\NormalTok{(}\SpecialCharTok{{-}}\DecValTok{1}\NormalTok{) }\SpecialCharTok{\%\textgreater{}\%}
\FunctionTok{chisq.test}\NormalTok{(.)}
\end{Highlighting}
\end{Shaded}

\begin{verbatim}
## 
##  Chi-squared test for given probabilities
## 
## data:  .
## X-squared = 11.355, df = 2, p-value = 0.003423
\end{verbatim}

Since the result states the p value is lower than the significant alpha
value I picked, I can reject the null hypothesis in favor of the
alternative. This means that the table is an accurate representation of
the relation between the smoking students and the percentage of
nationalities within the student body.

\#3. University salaries -yearly salaries of random 52 academic workers
at one of the U.S. Universities. -friend has been working there for the
past 10 years. He achieved his doctorate 12 years ago, and is an
associate professor in the Geology Department. -He earns \$30.000 a
year. Is it a fair salary or not?

\begin{Shaded}
\begin{Highlighting}[]
\NormalTok{salary }\OtherTok{\textless{}{-}}\NormalTok{ readr}\SpecialCharTok{::}\FunctionTok{read\_csv}\NormalTok{(}\StringTok{\textquotesingle{}data/salaries.csv\textquotesingle{}}\NormalTok{)}
\end{Highlighting}
\end{Shaded}

\begin{verbatim}
## Rows: 52 Columns: 6
\end{verbatim}

\begin{verbatim}
## -- Column specification --------------------------------------------------------
## Delimiter: ","
## chr (3): sx, rk, dg
## dbl (3): yr, yd, sl
\end{verbatim}

\begin{verbatim}
## 
## i Use `spec()` to retrieve the full column specification for this data.
## i Specify the column types or set `show_col_types = FALSE` to quiet this message.
\end{verbatim}

\begin{Shaded}
\begin{Highlighting}[]
\FunctionTok{spec}\NormalTok{(salary)}
\end{Highlighting}
\end{Shaded}

\begin{verbatim}
## cols(
##   sx = col_character(),
##   rk = col_character(),
##   yr = col_double(),
##   dg = col_character(),
##   yd = col_double(),
##   sl = col_double()
## )
\end{verbatim}

\hypertarget{a-create-a-model-to-predict-salaries-at-this-university.-tune-it-so-that-it-is-most-statisticaly-significant}{%
\subsection{a) Create a model to predict salaries at this University.
Tune it, so that it is most statisticaly
significant}\label{a-create-a-model-to-predict-salaries-at-this-university.-tune-it-so-that-it-is-most-statisticaly-significant}}

I will create a linear model to predict `Average Salary' using the
variables in the dataset.

\begin{Shaded}
\begin{Highlighting}[]
\NormalTok{average\_salary }\OtherTok{\textless{}{-}} \FunctionTok{read\_delim}\NormalTok{(}\StringTok{\textquotesingle{}data/salaries.csv\textquotesingle{}}\NormalTok{, }\AttributeTok{delim =} \StringTok{\textquotesingle{},\textquotesingle{}}\NormalTok{)}
\end{Highlighting}
\end{Shaded}

\begin{verbatim}
## Rows: 52 Columns: 6
\end{verbatim}

\begin{verbatim}
## -- Column specification --------------------------------------------------------
## Delimiter: ","
## chr (3): sx, rk, dg
## dbl (3): yr, yd, sl
\end{verbatim}

\begin{verbatim}
## 
## i Use `spec()` to retrieve the full column specification for this data.
## i Specify the column types or set `show_col_types = FALSE` to quiet this message.
\end{verbatim}

Since there is data about the salaries in the company and we know the
amount of workers(52) I came up with the following formula to find out
the average income.

\begin{Shaded}
\begin{Highlighting}[]
\NormalTok{workers}\OtherTok{\textless{}{-}}\DecValTok{52}
\end{Highlighting}
\end{Shaded}

average\_salary=total\_salaries÷workers We calculate the total budget of
salaries

\begin{Shaded}
\begin{Highlighting}[]
\NormalTok{(total\_salaries }\OtherTok{\textless{}{-}} \FunctionTok{sum}\NormalTok{(salary}\SpecialCharTok{$}\NormalTok{sl))}
\end{Highlighting}
\end{Shaded}

\begin{verbatim}
## [1] 1237478
\end{verbatim}

\begin{Shaded}
\begin{Highlighting}[]
\NormalTok{(average\_salary}\OtherTok{\textless{}{-}}\NormalTok{ total\_salaries }\SpecialCharTok{/}\NormalTok{ workers)}
\end{Highlighting}
\end{Shaded}

\begin{verbatim}
## [1] 23797.65
\end{verbatim}

The conclusion is that he earns more than average.

\end{document}
